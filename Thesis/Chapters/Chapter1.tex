% Chapter Template

\chapter{Introducció} % Main chapter title

L'objectiu d'aquest capítol és per una banda introduir els conceptes fonamentals per a la construcció dels grups $K_0$ i $K_1$, en especial $K_0$, i per l'altra exposar la propietat IBN com a plantejament del problema de la generalització del concepte de dimensió sobre $R$-mòduls. \\
El contingut està basat en {\normalfont [2]} per a les sumes directes, idemopotents, producte tensorial i propietat IBN, i {\normalfont [1]} per a la introducció a mòduls projectius, també hem usat {\normalfont [6]} per a provar que els $R$-mòduls sobre anells commutatius compleixen la propietat IBN.
\\
\indent Inicialment l'annex \ref{AppendixA} estava inclòs en aquesta secció, aquest annex té el fi de posar com a protagonista indiscutible del treball la $K$-teoria algebraica i al mateix temps permetre que el treball sigui autocontingut. \\
En aquest annex es recorden les definicions essencials que s'han vist durant el grau, i conceptes de mòduls, submòduls, mòduls quocients, mòduls lliures, bases i mòduls finitament generats, els quals un graduat recent no hi està necessàriament familiaritzat i que convé tenir presents durant la lectura del treball.
\label{Chapter1} % Change X to a consecutive number; for referencing this chapter elsewhere, use \ref{ChapterX}

\section{Sumes directes i Idempotents}
En aquesta secció presentem els conceptes fonamentals de suma directa, interna i externa. Aquests conceptes tenen un rol fonamental al caracteritzar els mòduls projectius, els quals són la font de $K_0$. També s'introdueix la relació entre els idempotents i la suma directa, que també usarem per a caracteritzar $K_0$.

\begin{definition}
Diem que $M$ és la \textbf{suma directa externa} $L \oplus N$ si $M$ és el producte cartesià $L\times N = \{(x,y) : x\in L, \ y\in N \}$ amb suma component a component $$(x_1,x_2)+(x_2,y_2)=(x_1+x_2,y_1+y_2)$$ i producte escalar $r(x,y)=(rx,ry)$.
\\
\indent Diem que $M$ és la suma directa interna $L\overset{\bullet}{\oplus} N$, si $L$ i $N$ són submòduls de $M$, $L+N=M$ i $L\cap N = {0_M}$.  
\\ \indent
En qualsevol anell $S$, un \textbf{idempotent} és un element $e\in S$, amb $ee=e$. 
\end{definition}

\begin{prop} \label{primer}
Suposem que $L$ i $N$ són submòduls de $M$. Són equivalents:
\begin{enumerate}[i)]
\item $M=L\overset{\bullet}{\oplus} N$
\item Cada $m\in M$ té una i només una expressió de la forma $m=x+y$ amb $x\in L$ i $y\in N$.
\item Existeix un idempotent $e\in \text{End}_R(M)$ amb imatge $L$ i nucli $N$. 
\end{enumerate}
\label{img+ker=M}
\end{prop}

\begin{proof}
En les condicions de $i)$, pel fet que $M=L+N$, tenim que tot $m\in M$ té una expressió $m=x+y$, amb $x\in L$ i $y\in N$. Suposem que existeix una expressió diferent $m=x'+y'$ amb $x'\in L$ i $y'\in N$, aleshores $x-x'=y'-y\in L \cap N = \{0_M\}$, per tant $x=x'$ i $y=y'$ provant la unicitat de l'expressió. 
\\
\indent En les condicions de $ii)$, a d'existir una aplicació $R$-lineal \\ $e: M \rightarrow M$, tal que $e(x+y)=x$ si $x\in L$ i $y\in N$. Observem que al complir-se que $$(e\circ e)(x+y)=e(x)=e(x+0)=x=e(x+y)$$ queda comprovat que $e$ és un idempotent, amb imatge $L$ i nucli $N$.
\\
\indent En les condicions de $iii)$, per cada $m\in M$, $m=e(m)+m-e(m)$ amb $e(m)\in L$ i $m-e(m)\in \text{ker}(e)=N$. Si $x\in L\cap N$, aleshores $x=e(y)$ per algun $y\in M$, i $$0_M=e(x)=e(e(y))=e(y)=x.$$ Per tant, $M=L\overset{\bullet}{\oplus}N$.
\end{proof}

\begin{prop}
Siguin $L$, $M$ i $N$ $R$-mòduls. Aleshores $M\cong L \oplus N$ si i només si $M=L' \overset{\bullet}{\oplus} N'$ per $L'$, $N'$ submòduls de $M$ tals que $L'\cong L$ i $N' \cong N$.
\end{prop}

\begin{proof}
Si $f: L \oplus N \rightarrow M$ és un isomorfisme, tenim que $L\cong L \oplus 0 \cong f(L\oplus 0)$, $N\cong 0 \oplus N \cong f(0\oplus N)$, i 
$$
M=f(L\oplus 0)\overset{\bullet}{\oplus} f(0 \oplus N).
$$
Recíprocament, si $\alpha: L \rightarrow L'$, $\beta: N \rightarrow N'$ són isomorfismes i $M=L' \overset{\bullet}{\oplus} N'$, aleshores
\begin{eqnarray*}
        f: L \oplus N& \rightarrow M\\
        (x,y)& \mapsto  \alpha (x)+\beta (y)
\end{eqnarray*}
 és $R$-lineal i bijectiva per la proposició anterior.
\end{proof}
\section{Producte tensorial}

Si $K$ és un cos, i $J$ el conjunt de classes d'isomorfisme de $K$-espais vectorials, la dimensió sobre $K$ defineix un isomorfisme de monoides $$(J,\oplus) \cong (\mathbb{N},+).$$
\hspace*{.5cm}Tanmateix $\mathbb{N}$ és un semianell sota les operacions suma i producte, usant l'isomorfisme de dimensió podem donar estructura de semianell a $J$, és a dir si $V$ i $W$ són $K$-espais vectorials de dimensió finita, ha d'existir un espai vectorial $X$ amb $c(V)c(W)=c(X)$ a $J$ amb $\text{dim}_FX = (\text{dim}_FV)(\text{dim}_FW)$. \\
\hspace*{.5cm}Per a la construcció de $X$ a partir de $V$ i $W$ necessitem una forma de multiplicar elements d'un mòdul per elements d'un altre mòdul. 

\begin{definition}
Si $R$ és un anell, donats dos $R$-mòduls $M_R$, $_R{N}$ i un grup abelià additiu $A$, l'aplicació $f:M\times N \rightarrow A$ és \textbf{$R$-equilibrada} si compleix:
\begin{eqnarray*}
f((m+m',n))=f((m,n))+f((m',n)) \\
f((m,n+n'))=f((m,n))+f((m,n')) \\
f((mr,n))=f((m,rn))
\end{eqnarray*}
per a tot $m,m'\in M$, $n,n'\in N$ i $r\in R$.

\end{definition}

\indent Hi ha una aplicació universal $R$-equilibrada $M_r \times _R{N} \rightarrow A$ tal que el seu codomini $A$ és el producte de mòduls que busquem, ara la construïm. 

\begin{definition}
 Si $R$ és un anell, el \textbf{producte tensorial} dels mòduls $M_R$ i $_R{N}$ sobre l'anell $R$ és el grup quocient
$$
M \otimes_R N := F_{\mathbb{Z}}(M\times N)/\langle D \rangle 
$$
on  $F_{\mathbb{Z}}(M\times N)/\langle D \rangle $ és el $\mathbb{Z}$-mòdul lliure construït per tenir com a base el conjunt $M\times N$, i $D$ el subgrup generat per elements que tinguin una de les següents tres formes:
\begin{eqnarray*}
(m+m',n) - (m,n) - (m',n), \\
(m,n+n') - (m,n) - (m.n'), \\
(mr,n)-(m,rn),
\end{eqnarray*}
per $m,m'\in M$, $n,n'\in N$, i $r\in R$. Si $m\in M$ i $n\in N$, denotem per $m\otimes n:=(m,n)+\langle D \rangle$.\\
\hspace*{.5cm} Per construcció $M\otimes_R N$ és un grup abelià additiu amb generadors $m\otimes n$ i relacions:
\begin{eqnarray*}
(m+m')\otimes n &=& (m \otimes n) + (m'\otimes n),\\
m\otimes(n+n') &=& (m \otimes n) + (m \otimes n'),\\
mr\otimes n &=& m \otimes rn
\end{eqnarray*}
per $m,m'\in M$, $n,n'\in N$ i $r\in R$. L'aplicació natural 
\begin{eqnarray*}
t: M\times N &\rightarrow &M\otimes_R N \\
(m,n) &\mapsto & m\otimes n
\end{eqnarray*}
és equilibrada, de fet, és l'aplicació amb codomini $M\otimes_R N$ que volíem construir.
\\
\end{definition}

\begin{prop}
Si $R$ és un anell amb mòduls $M_R$ i $_R{N}$, aleshores per a cada funció $R$-equilibrada $f:M\times N \rightarrow A$, hi ha un i només un homomorfisme de grups additius:
$$
\overline{f}:M\otimes_R N \rightarrow A \hspace{1cm} \text{amb}\hspace{1cm}\overline{f}(m\otimes n) = f((m,n))
$$ 
per a tot $m\in M$ i $n\in N$; aquest és el únic homomorfisme de grups $\overline{f}$ tal que fa commutar el triangle:

\begin{center}
	\begin{tikzpicture}
  \matrix (m) [matrix of math nodes,row sep=4em,column sep=4em,minimum width=2em]
  {
     M\times N & M \otimes_R N \\
      & A \\};
  \path[->]
  	(m-1-1.east|-m-1-2) edge node [above] {$t$}
            node [above] {} (m-1-2)
	(m-1-2) edge node [right] {$\overline{f}$} (m-2-2)
	(m-1-1) edge node [left] {$f$} (m-2-2);
    %edge [dashed,-] (m-2-1);
	\end{tikzpicture}
\end{center}
\end{prop}

\begin{proof}
El fet que $f$ sigui $R$-equilibrada fa que la seva extensió $\mathbb{Z}$-lineal $F_{\mathbb{Z}}(M\times N)$ enviï $D$ a $0$, per tant l'existència de $\overline{f}$ ve induïda per $\overline{f}(m\otimes n) = f((m,n))$. La unicitat és conseqüència del fet que els elements de la forma $m\otimes n$ generen $M\otimes_RN$.
\end{proof}

\begin{definition}
Sigui $R$ i $S$ anells. Diem que un grup abelià additiu $M$ és un \textbf{bimòdul} si és un $R$-mòdul per l'esquerra i un $S$-mòdul per la dreta, i ha de complir $r(ms)=(rm)s$, el denotem $_R{M_S}$.
\end{definition}

\begin{prop}
 Donats tres anells $R$,$S$,$T$ i dos bimòduls $_R{M_S}$ i $_S{N_T}$, el producte tensorial sobre $S$ és un bimòdul $_R{M \otimes N}_T$. 
\end{prop}

\begin{proof}
Si $r\in R$, la funció 
\begin{eqnarray*}
f_r : M\times N &\rightarrow & M\otimes_S N\\
(m,n) & \mapsto & (rm \otimes n)
\end{eqnarray*}
és $S$-equilibrada.
\\
\hspace*{.5cm}Observem que a la definició de bimòdul anterior, demanem per definició que un bimòdul compleixi $r(ms)=(rm)s$, aquesta propietat és necessària per a que la funció anterior sigui $S$-equilibrada, en particular, és necessària per a que compleixi $f_r((ms,n))=f_r((m,sn))$. \\
\\
Per tant, usant la proposició anterior hi ha un homomorfisme de grups $\mathbb{Z}$-lineal
$$
\overline{f}_r: M \otimes_S N \rightarrow M \otimes_S N
$$
amb $\overline{f}_r(m\otimes n)=(rm) \otimes n$, per $m\in M$ i $n\in N$. Observem que la funció
\begin{eqnarray*}
R\times(M\oplus_SN) & \rightarrow M & \otimes_S N \\
(r.x) & \mapsto & r \cdot x = \overline{f}_r(x)
\end{eqnarray*}
és el producte per escalars que dóna estructura de $R$-mòdul per l'esquerra a $M\otimes_S N$, ja que com que $\overline{f}_r$ és $\mathbb{Z}$-lineal i

\begin{eqnarray*}
(r+r')\cdot (m\otimes n) &=& ((r+r')m \otimes n)) = (rm+r'm \otimes n)=(rm \otimes n) + (r'm \otimes n),\\
(rr')\cdot (m\otimes n) &=&r\cdot (r' \cdot  (m\otimes n)),\\
1\cdot(m\otimes n) &=& m\otimes n
\end{eqnarray*}
Sota el producte per escalars, tenim
$$
r \cdot \sum_i^k (m_i \otimes n_i) = \sum_{i=1}^{k}(rm_i)\otimes n_i
$$
Similarment, $M\otimes_S N$ és un $T$-mòdul per la dreta via 
$$
\left( \sum_{i=1}^k (m_i\otimes n_i) \right) \cdot t = \sum_{i=1}^k m_i \otimes (n_i t)
$$
Finalment per a provar que $M \otimes_S N$ és un $R,T$-bimòdul, només cal veure que 
$$
r\cdot \left(\sum (m_i \otimes n_i) \cdot t \right) = \sum (rm_i) \otimes (n_it) = (r\cdot \sum (m_i \otimes n_i)) \cdot t.
$$
\end{proof}


%https://en.wikipedia.org/wiki/Tensor_product_of_modules#Properties
\section{La propietat IBN}

Tot espai vectorial sobre un cos té una base, i dues bases del mateix espai vectorial tenen sempre el mateix nombre d'elements, per tant el concepte de dimensió d'un espai vectorial $V$ està ben definit. A més la dimensió és un invariant que classifica els espais vectorials. \\
Hom podria pensar que aquest resultat es manté al generalitzar el concepte d'espai vectorial sobre un cos a espai vectorial sobre un anell, i.e., sobre $R$-mòduls, però no és així.\\
\indent En un $R$-mòdul $M$, l'existència d'una base no està garantida, i en cas de tenir base, podem tenir bases amb diferent cardinal, i.e., el cardinal de la base no és un invariant. L'objectiu d'aquesta subsecció és comprovar aquests fets.
\\\\
\indent La millor forma de provar que l'existència de la base no està garantida és trobar un $R$-mòdul que no tingui base.
\begin{prop}
Els grup additiu format pels racionals $\mathbb{Q}$ és un $\mathbb{Z}$-mòdul sense base.
\end{prop}
\begin{proof}
En primer lloc observem $\mathbb{Q}$ no és un $\mathbb{Z}$-mòdul cíclic, i.e., no pot estar generat per un únic element. D'altra banda, observem que dos elements qualsevols de $\mathbb{Q}$ són $\mathbb{Z}$-linealment dependents, ja que donats $m,n\in \mathbb{Q}$, existeixen $a,b\in \mathbb{Z}$ tal que es forma una combinació $\mathbb{Z}$-lineal amb $am+bn=0$. \\
Així doncs aquest $\mathbb{Z}$-mòdul no pot tenir una base formada per un element, ni una base formada per més d'un element.
\end{proof}

\begin{definition}
Diem que un anell $R$ compleix la propietat \textbf{IBN} (\textbf{I}nvariant \textbf{B}asis \textbf{N}umber) si per a tot $R$-mòdul $M$, tota parella de bases de $M$ té el mateix cardinal, és a dir, el nombre d'elements bàsics és invariant.
\end{definition}

Veiem ara que el cardinal de la base no és un invariant sobre $R$-mòduls, en particular trobarem un anell sense la propietat IBN, i.e. un $R$-mòdul amb dues bases de cardinal diferent.

\begin{prop}
El cardinal de la base no és un invariant sobre tot $R$-mòdul, amb $R$ anell no necessàriament commutatiu.
\end{prop}
\begin{proof}
Per a qualsevol $R$ anell no trivial, considerem $A$ com el conjunt de matrius definides sobre $R$ amb infinites files i columnes, tals que cada columna tingui un nombre finit d'entrades diferents de zero. \\
Observem que la suma i producte habitual doten $A$ amb l'estructura d'anell. A més, si considerem el conjunt de columnes d'elements de $A$, tenim un $A$-mòdul. En concret si $A\in A$, la columna $j$-èssima de $A$ la denotarem per $ae_j$, al considerar $A$ com a $A$-mòdul tenim que la suma i el producte són calculats per columnes:
\begin{eqnarray*}
(a+b)e_j&=ae_j+be_j \\
(ab)e_j &= a(b e_j)
\end{eqnarray*}

Així doncs tenim que $A$ és un $A$-mòdul. Per una banda observem que a la matriu identitat, denotada per 
$$1_{A}=\begin{bmatrix}
1  & 0 & \hdots  \\
0  & 1 & \hdots \\
\vdots & \vdots & \ddots
\end{bmatrix}, 
$$
és una base de $A$, i.e., $A$ té una base formada per un element. \\
\indent D'altre banda, per a cada $a\in A$, definim $a', a''$ de forma que les columnes de $a'$ siguin les columnes senars de $a$, i les columnes de $a''$ siguin les columnes parells de $a$. \\
Aleshores la funció $f: A \rightarrow A \oplus A$ definida per $f(a)=(a', a'')$ és un isomorfisme $A$-lineal amb inversa formada per ajuntar les columnes de $a'$ i $a''$. \\
Usant el fet que la imatge d'elements bàsics és element bàsic, podem veure el resultat al Corol·lari \ref{corol1}, tenim que $A$ té una base formada per dos elements

$$
f^{-1}(1_A,0_A)=\begin{bmatrix}
1  & 0 & 0 & 0 & \hdots  \\
0  & 0 & 1 & 0 & \hdots \\
\vdots & \vdots & \vdots & \vdots & 
\end{bmatrix}
, \hspace{1cm}
f^{-1}(0_A,1_A)=\begin{bmatrix}
0  & 1 & 0 & 0 & \hdots  \\
0  & 0 & 0 & 1 & \hdots \\
\vdots & \vdots & \vdots & \vdots & 
\end{bmatrix}
$$
\end{proof}


Finalment veurem que tot $R$ anell commutatiu no trivial  compleixen la propietat IBN, per a veure aquest resultat veiem el següent Lema.

\begin{lemma}
Sigui $R$ un anell commutatiu i siguin $m,n$ dos enters qualsevols tals que $m<n$, si hi ha una aplicació lineal exhaustiva $R^m \rightarrow R^n$. Aleshores $1_R=0_R$.
\end{lemma}
\begin{proof}
Sigui $\varphi$ l'aplicació exhaustiva que envia $R^m \rightarrow R^n$. L'exhaustivitat de $\varphi$ garanteix l'existència d'una aplicació $\psi$ \textit{(inversa per la dreta)} tal que $\varphi \circ \psi = \id$. Podem extendre $\varphi$ a $R^n = R^m \oplus R^{n-m}$ de forma que $\varphi(R^{n-m})=0_R$, i considerar $\psi$ com una aplicació sobre $R^n$. \\
Considerem ara les matrius $A$ i $B$ associades a les aplicacions $\varphi$ i $\psi$ respectivament. Per una banda tenim que $\det (AB) = 1_R$, d'altre banda tenim que $ \det (B)=0$ ja que la darrera fila de $B$ és plena de $0_R$'s per construcció. \\ Usant la propietat $\det(AB)=\det(A)\det(B)$ obtenim el resultat que buscàvem.
\end{proof}
Observem que a la prova anterior hem pogut usar l'existència del determinant gràcies al fet que $A$ i $B$ són matrius definides sobre un anell commutatiu.

\begin{corollary}
Si $R$ és un anell commutatiu no trivial, compleix la propietat IBN, i.e., els $R$-m.ll.f.g. tenen un nombre d'elements bàsics invariant. \footnote{Usem l'abreviatura per $R$-mòdul lliure finitament generat ($R$-m.ll.f.g.).}
\end{corollary}


%%%%%%%%%%%%%%%%%%%%%%%%%%%%%%%%%%%%%%%%%%%%%%%%%%%%%%%%%%%%%
%%%%%%%%%%%%%%%%%%%%%%MÒDULS PROJECTIUS%%%%%%%%%%%%%%%%%%%%%%
%%%%%%%%%%%%%%%%%%%%%%%%%%%%%%%%%%%%%%%%%%%%%%%%%%%%%%%%%%%%%

\section{Mòduls projectius}

Els mòduls projectius tenen un rol fonamental en la construcció de $K_0$, en aquesta secció introduïm la definició dels mòduls projectius i provem la caracterització fonamental dels mòduls projectius usant la suma directa.

\begin{definition}
 Donat un anell $R$, anomenem \textbf{mòdul projectiu sobre $R$} a un $R$-mòdul $P$ si compleix que, per a tot homomorfisme de $R$-mòduls exhaustiu $\alpha: M \twoheadrightarrow P$ existeix un homomorfisme $\beta: P \rightarrow M$ tal que $\alpha \circ \beta = id_P$, és a dir, tot homomorfisme de $R$-mòduls exhaustiu té una inversa per la dreta.
\end{definition}

%%%%%%%%%%%%%%%%%%%%%%%%%%%%%%%%%%%%%%%%%%%%%%%%%%%%%%%%%%%%%
\begin{lemma}
Un $R$-mòdul $P$ és projectiu si i només si tot diagrama de $R$-mòduls com el de la Figura \ref{DR-M} amb $\psi$ exhaustiva
%---------------------------------------------------
\begin{figure}[!htb]
\minipage{0.32\textwidth}
\begin{flushright}
 	\begin{tikzpicture}
  \matrix (m) [matrix of math nodes,row sep=3em,column sep=4em,minimum width=2em]
  {
      & P \\
     M & N \\};
  \path[->>]
    (m-2-1.east|-m-2-2) edge node [below] {$\psi$}
            node [above] {} (m-2-2);
    \path[->](m-1-2) edge node [right] {$\phi$} (m-2-2);
   % \draw[dashed,->] (m-1-2) -- (m-2-1);
\end{tikzpicture}
\caption{}
\label{DR-M}
\end{flushright}
\endminipage\hfill
\minipage{0.32\textwidth}
\textit{es pot completar fent-lo commutatiu, i.e., existeix una aplicació $\theta$ tal que} 


\endminipage\hfill
\minipage{0.32\textwidth}%
	\begin{tikzpicture}
  \matrix (m) [matrix of math nodes,row sep=3em,column sep=4em,minimum width=2em]
  {
      & P \\
     M & N \\};
 \path[->>]
    (m-2-1.east|-m-2-2) edge node [below] {$\psi$}
            node [above] {} (m-2-2);
    \path[->](m-1-2) edge node [right] {$\phi$} (m-2-2);
   \draw[dashed,->] (m-1-2) edge node [left] {$\theta$} (m-2-1);
\end{tikzpicture}
\endminipage
\end{figure}
%--------------------------------------------------
\end{lemma}
%%%%%%%%%%%%%%%%%%%%%%%%%%%%%%%%%%%%%%%%%%%%%%%%%%%%%%%%%%%%%%%%%%%%%%%%%

\begin{proof} Suposem que es compleix la propietat de completació de diagrames, sigui $\alpha$ un  homomorfisme exhaustiu de $R$-mòduls $\alpha: M\rightarrow P$, hem de veure que té una inversa per la dreta. podem prendre $N=P$, $\phi = id_P$, i $\psi = \alpha$. Aleshores el $\theta:P \rightarrow M$ resultant de la propietat de completació de diagrames compleix $\alpha \circ \theta = id_P$, i.e. $\beta$ és la inversa per la dreta de $\alpha$.
\\
\indent Veiem ara el recíproc, suposem que tot homomorfisme de $R$-mòduls exhaustiu $\alpha : M \rightarrow P$ té una inversa per la dreta $\beta : P \rightarrow M$. Donat un diagrama de $R$-mòduls com el de la Figura \ref{DR-M} , canviant $M \xrightarrow{\psi} N$ per $M \oplus P \xrightarrow{\psi \oplus id_P} N \oplus P$ i $\phi:P\rightarrow N$ per $((\phi, id_P):P \rightarrow N\oplus P)$, podem suposar que $\phi$ és injectiva (Figura \ref{finj}). A més, si canviem $N$ per la imatge de $\phi$ i $M$ per $\psi^{-1}(\phi(P))$, podem suposar que $\phi$ és un isomorfisme (Figura \ref{fbij}). Aleshores, considerant $\beta$ com la inversa per la dreta de $\alpha = \phi^{-1} \circ \psi$ tenim que $\beta : P \rightarrow M$ ens permet completar el diagrama.






%------------------------------------------------
%---------------------------------------------------
\begin{figure}[!htb]
\minipage{0.22\textwidth}

\begin{center}
 	\begin{tikzpicture}
  \matrix (m) [matrix of math nodes,row sep=3em,column sep=4em,minimum width=2em]
  {
      & P \\
     M & N \\};
  \path[->>]
    (m-2-1.east|-m-2-2) edge node [below] {$\psi$}
            node [above] {} (m-2-2);
    \path[->](m-1-2) edge node [right] {$\phi$} (m-2-2);
   % \draw[dashed,->] (m-1-2) -- (m-2-1);
\end{tikzpicture}
\end{center}
\endminipage\hfill
\minipage{0.32\textwidth}
\begin{center}
	\begin{tikzpicture}
  \matrix (m) [matrix of math nodes,row sep=3em,column sep=4em,minimum width=2em]
  {
      & P \\
     M \oplus P& N \oplus P \\};
  \path[->>]
    (m-2-1.east|-m-2-2) edge node [below] {$\psi \oplus id_P$}
            node [above] {} (m-2-2);
    \path[->](m-1-2) edge node [right] {$\phi \oplus id_P$} (m-2-2);
   % \draw[dashed,->] (m-1-2) -- (m-2-1);
\end{tikzpicture}
\caption{} \label{finj}
\end{center}

\endminipage\hfill
\minipage{0.42\textwidth}%
\begin{center}
\begin{tikzpicture}
  \matrix (m) [matrix of math nodes,row sep=3em,column sep=4em,minimum width=2em]
  {
      & P \\
     \psi^{-1}(\phi(P)) \oplus P & \phi(P) \oplus P \\};
  \path[->>] (m-2-1) edge node [below] {$\psi \oplus id_P$}(m-2-2)
            node [above] {} (m-2-2);
            
   \path[->] (m-1-2) edge node [right] {$\phi \oplus id_P$} (m-2-2);
    %edge [dashed,-] (m-2-1);
\end{tikzpicture}
\caption{} \label{fbij}
\end{center}

\endminipage
\end{figure}
%--------------------------------------------------
%------------------------------------------------

\end{proof}

%%%%%%%%%%%%%%%%%%%%%%%%%%%%%%%%%%%%%%%%%%%%%%%%%%%%%%%%%%%%%%%%%%%%%%

\begin{obs}\label{sumasplit}Observem que si $\alpha : M\rightarrow P$ és exhaustiva i $\beta:P\rightarrow M$ és la seva inversa per la dreta, aleshores $p:=\beta \circ \alpha$ és un endomorfisme idempotent de $M$ ja que 
\begin{equation} 
\begin{split}
(\beta \circ \alpha)^2 & = (\beta \circ \alpha) \circ (\beta \circ \alpha) \\
 & = \beta \circ (\alpha \circ \beta) \circ \alpha \\
 &= \beta \circ id_P \circ \alpha  = \beta \circ \alpha
\end{split}
\end{equation}
A més, usant la Proposició \ref{img+ker=M} tenim que $M \cong P \oplus \text{Ker($p$)} = P \oplus (1-p)(M) $
\end{obs}

\begin{lemma}
 Sigui $R$ un anell, tot $R$-mòdul lliure és projectiu. 
\end{lemma}
\begin{proof} Sigui $F$ un $R$-mòdul lliure, al ser lliure ha de tenir una base $B$. Sigui $\alpha: M \rightarrow F$ un homomorfisme de mòduls exhaustiu,  gràcies a l'exhaustivitat de $\alpha$ per a cada element bàsic $x_i\in B \subset F$ ha d'existir algun $y_i\in M$ tal que $\alpha(y_i)=x_i$. Definim l'aplicació $\beta$ a partir de la imatge dels elements bàsics $\beta(x_i)=y_i$, usant la Proposició \ref{morfismebase} $\beta$ és un morfisme $F\rightarrow M$ ben definit. A més, és inversa per la dreta de $\alpha$.
\end{proof}

\begin{theorem}[Caracterització fonamenteal dels mòduls projectius] \label{carProj}
Sigui $R$ un anell. Un $R$-mòdul és projectiu si i només si és isomorf a un sumand directe d'un $R$-mòdul lliure\footnote{i.e. existeix un $R$-mòdul $Q$ i un $R$-mòdul lliure $F$ tal que $F = P \oplus Q$.}. És finitament generat i projectiu si i només si és isomorf a un sumand directe de $R^n$ per algun $n$.
\end{theorem}

\begin{proof}Si $P$ és un mòdul projectiu, ha de tenir almenys un conjunt generador $G$ (no és necessàriament una base de $P$). Triem un $R$-mòdul $F$ amb base $B$ que ens permeti definir una aplicació bijectiva entre elements de $B$ i $G$. Usant de nou \ref{morfismebase} tenim que $\alpha$ té una extensió a una aplicació $R$-lineal $\hat{\alpha}: F\rightarrow P$. De fet, com que la imatge de $B$ és $P$, $\hat{\alpha}$ és exhaustiva. Usant l'Observació anterior tenim que $P$ és isomorf a una suma directa en un $R$-mòdul lliure, a més, si $P$ és finitament generat, podem escollir que $F$ sigui exactament $R^n$ on $n$ és el nombre d'elements de $G$.
\\
Per a veure el recíproc, suposem que $F\cong P\oplus Q$, on $F$ és un mòdul lliure. Donat un homomorfisme de $R$-mòduls exhaustiu $\alpha : M \rightarrow P$, observem que l'homomorfisme de $R$-mòduls $\alpha \oplus id_Q:(M\oplus Q)\rightarrow (P\oplus Q)=F$ també és exhaustiu . A més, com que $F$ és lliure, ha de ser projectiu. Així doncs, usant la projectivitat de $F$ tenim que $\alpha \oplus id_Q$ té una inversa per la dreta. Si restringim la inversa per la dreta a $P$ i la projectem sobre $M$ tenim una inversa per la dreta de $\alpha$. Per acabar, si $F\cong R^n$ amb generadors $x_1,\dots, x_n \in B$, aleshores $P$ ve generat pels $n$ elements $p(x_i)$, on $p$ és la identitat a $P$ i $0$ a $Q$.
\end{proof}
