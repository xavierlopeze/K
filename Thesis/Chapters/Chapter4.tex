
% Chapter 1

\chapter{Conclusions} 
El contingut d'aquest treball és només una petita part del que es coneix per $K$-teoria algebraica. Hem introduït els dos primers grups, els quals formen part del que es coneix per $K$-teoria algebraica clàssica. \\
\indent Per ampliar aquest treball en el sentit algebraic podríem haver estudiat $K_0$ i $K_1$ per dominis de Dedekind, aquesta estructura algebraica permet mostrar resultats interessants sobre $K$-grups. També podríem haver estudiat $K_0$ i $K_1$ per a categories en lloc d'anells, i introduir la $K$-teoria negativa. O fins i tot seria interessant estudiar el grup $K_2$. \\
\indent D'altra banda, és molt usual parlar de $K$-teoria en sentits no algebraics, cal destacar la $K$-teoria topològica, que és la branca de la topologia que es dedica a estudiar fibrats vectorials sobre espais topològics.  Ampliacions d'aquest treball en el sentit topològic poden ser interessants. \\
\indent Molts problemes profunds en àlgebra, com la classificació de subgrups normal de grups lineals, o la descripció del grup de Brauer d'un cos en termes d'àlgebres cícliques han estat resolts a través de $K$-teoria algebraica. Possibles ampliacions d'aquest treball podrien estar enfocades a estudiar aquests resultats en detall.\\
\indent Però més enllà d'això, la $K$-teoria algebraica ha aportat tècniques de resolució de problemes importants en topologia, geometria, teoria de nombres i anàlisi funcional.
En aquest treball ens hem dedicat a introduir els fonaments bàsics d'aquestes tècniques, la punta del iceberg.